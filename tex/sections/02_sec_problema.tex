\section{O (suposto) Problema}

\begin{frame}{Teólogos que veem problema}
\begin{columns}
\begin{column}{0.7\textwidth}
	 \begin{itemize}
			 \item<2->[$\bullet$] \textcolor{NordBrightCyan}{\textbf{Gilkey}}: teólogos não acreditam que Deus 
				 tenha realizado milagres!
				 \begin{itemize}
					 \item<5-> a Bíblia é um livro de interpretação
					 \item<6-> dizer $\times$ crer
				 \end{itemize}
			 \item<3->[$\bullet$] \textcolor{NordBrightCyan}{\textbf{Bultmann}}: Deus não age no mundo!
			  \begin{itemize}
					 \item<7-> a história é uma \textcolor{NordYellow}{sequência fechada de efeitos}.
					 \item<8-> não pode sofrer intervenções de poderes sobrenaturais
					 \item<9-> causa e efeito
					 \item<10-> Lei dos Medos e Persas
				 \end{itemize}
			 \item<4->[$\bullet$] \textcolor{NordBrightCyan}{\textbf{Macquarrie}}: rompimento da ordem natural
			  \begin{itemize}
					 \item<11-> ciência e história são \textbf{\textcolor{NordRed}{inconciliáveis}}
					  com a ideia de ``milagres''.
					 \item<12-> motivo: causa e efeito
					  \begin{itemize}
					   \item<13->[-] conhecimento limitado apenas temporariamente.
						\end{itemize}
				\end{itemize}
		\end{itemize}
\end{column}
\begin{column}{0.3\textwidth}
 \only<2, 5-6>{\includegraphics[width = 4cm]{gilkey}}
	\only<3, 7-10>{\includegraphics[width = 4cm]{bultmann}}
	\only<4, 11-13>{\includegraphics[width = 4cm]{macquarrie}}
\end{column}
\end{columns}
\end{frame}

\begin{frame}{Teólogos que veem problema}
\centering
\begin{minipage}{\textwidth}
	\begin{exampleblock}{\textbf{Observações}}
	 \begin{itemize}
			\item<2->[$\bullet$] Bultmann e Macquarrie acham compatíveis a ``ação geral'' de Deus 
			 (preservação da existência do mundo), mas não a ``ação particular'';
			\item<3->[$\bullet$] A ação particular é um problema porque: é incompatível com a ciência
			 moderna.
				\begin{itemize}
					\item<4-> a ciência \textcolor{NordYellow}{demonstra} ou 
					 \textcolor{NordYellow}{pressupõe} que Deus não age assim;
					\item<5-> a ciência é a Razão.
				\end{itemize}
		\end{itemize}
	\end{exampleblock}
	\end{minipage}
\end{frame}

\begin{frame}{Filósofo ou Cientistas que veem problema}
 \begin{columns}
	 \begin{column}{0.45\textwidth}
	  \begin{itemize}
			 \justifying
		  \item<2-3,8-14>[$\bullet$] \textcolor{NordBrightCyan}{Philip Clayton}: a ciência 
				 tem a capacidade de explicar e prever os fenômenos naturais.
	  \end{itemize}
			\vspace{0.5cm}
			\centering
			\only<2-3, 8-14>{\includegraphics[width = 4cm]{clayton}}
			\only<4>{\includegraphics[width = 4cm]{dawkins}}
			\only<5>{\includegraphics[width = 4cm]{peter}}
			\only<7>{\includegraphics[width = 4cm]{orr}}
			\only<6>{\includegraphics[width = 4cm]{vaca}}
			\only<15>{\includegraphics[width = 4cm]{merovingian}}
			\only<16>{\includegraphics[width = 4cm]{sherlock}}
			\only<17>{\includegraphics[width = 4cm]{watson-skinner}}
		\end{column}
		\begin{column}{0.55\textwidth}
		 \begin{itemize}
			 \item<3-> Cientitas com Clayton: 
				 \begin{itemize}
					 \item<4-> \onslide<4->{Richard Dawkins}\onslide<5->{, Peter Atkins} \onslide<6->{(loucos?)}
					 \item<7-> H. Allen Orr (sensato)
					\end{itemize}
				\item<8-|alert@8> Ciência $\Longleftrightarrow$ \textcolor{NordYellow}{Determinismo}
					\begin{itemize}
						\item<9-> Exclusão de Deus (ação particular)
						\item<10-> At 17.18: estóicos \textit{vs} epicureus
							\begin{itemize}
								\item<11-> virtude: reação diante do determinismo materialista
								\item<12-> o ser humano não pode determinar seu destino
								\item<13-> ataraxia
								\item<14-> Síndrome de Gabriela
							\end{itemize}
						\item<15-> Filme Matrix
						\item<16-> Sherlock Holmes
						\item<17-> Skinner e Watson
					\end{itemize}
			\end{itemize}
		\end{column}
	\end{columns}
\end{frame}

\begin{frame}{Um dos Perigos do Determinismo}
 \begin{quote}
		\onslide<2->{Nossos membros estão praticamente sempre fazendo o que 
		\textcolor{NordOrange}{querem} fazer --- o que eles 
		\textcolor{NordOrange}{escolhem} fazer --- mas nós cuidamos para que eles 
		queiram fazer precisamente as coisas	que são melhores para eles e para a 
		comunidade.}
		\onslide<3->{Seu comportamento é determinado, ainda que eles sejam livres.}
		\onslide<4->{Ditadura e liberdade, determinismo e livre-arbítrio.}
		\onslide<5->{O que é isso senão pseudoquestões de origem linguística?}
		\onslide<6->{Quando perguntamos o que o Homem pode fazer do Homem, nós não 
		queremos dizer	a mesma coisa por ``homem'' em ambos os casos.}
		\onslide<7->{Queremos perguntar o que alguns poucos homens podem fazer da 
		humanidade.}
		\onslide<8->{E essa é a questão central do século XX.} 
		\onslide<9->{Que tipo de mundo podemos construir --- nós que entendemos a 
		ciência do comportamento?} \onslide<10->{(\textcolor{NordYellow}{Skinner})}
	\end{quote}
\end{frame}

\begin{frame}{\textbf{Resumo do Problema}}
	\centering
			\begin{minipage}{\textwidth}
			\begin{exampleblock}{\textbf{Resumo do ``Problema''}}
			 \begin{enumerate}[(1)]
					\item<2-> A Ciência descobre e endossa Leis Naturais;
					\item<3-> As ações particulares de Deus violariam as Leis Naturais;
					\item<4-> Isso é incompatível com a Ciência.
				\end{enumerate}
			\end{exampleblock}
			\end{minipage}
\end{frame}