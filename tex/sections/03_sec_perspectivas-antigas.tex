\section{Perspectivas Antigas}

%------------------------------------------------------------------------------
\subsection{Panorama}
%------------------------------------------------------------------------------
\begin{frame}{\textbf{Panorama}}
 \begin{itemize}
  \item<2->[$\bullet$] Tal conflito deriva de uma \textcolor{NordRed}{concepção particular da ciência clássica.}
   \begin{itemize}
    \only<3>{\item \textit{Weltanshauung}: visão de mundo}
    \item<4-> mecânica newtoniana
    \item<5-> física da eletricidade e magnetismo
     \begin{itemize}
      \item<6->[-] conservação do momento (3ª Lei de Newton)
      \item<7->[-] conservação da energia
     \end{itemize}
  \end{itemize}
 \end{itemize}
\end{frame}

%------------------------------------------------------------------------------
\subsection{A perspectiva newtoniana}
%------------------------------------------------------------------------------
\begin{frame}{\textbf{A perspectiva newtoniana}}
  \centering
  \begin{minipage}{\textwidth}
    \begin{exampleblock}<2->{O que diz?}
    O Universo material é uma imensa máquina que evolui ou opera de acordo com
    \textcolor{NordYellow}{leis fixas} (leis da Física Clássica)
    \end{exampleblock}
  \end{minipage}
	
  \vspace{0.5cm}
  \begin{itemize}
    \item<3->[$\bullet$] As leis refletem a natureza das coisas
    \item<4->[$\bullet$] Podem ser decretos de Deus para o comportamento da 
                         matéria
    \item<5->[$\bullet$] Todo mundo mecânico poderia ser reduzido às leis 
                         físicas
     \begin{itemize}
       \item<6-> inclusive Leis da Biologia ou da Química
       \item<7-> \textcolor{NordRed}{completude da física clássica}: um 
                                     acréscimo filosófico
     \end{itemize}
  \end{itemize}
\end{frame}

\begin{frame}{\textbf{A perspectiva newtoniana}}
  \begin{itemize}
    \item<2->[$\bullet$] \onslide<2->{Perspectiva newtoniana} 
     \onslide<3->{$\not\Rightarrow$} 
     \onslide<4->{\textcolor{NordYellow}{teologia da não interferência divina}}
     \begin{itemize}
      \item<5-> Newton aceitava a interferência divina: ajustes das órbitas dos 
       planetas
      \item<6-> As Leis Naturais descrevem o funcionamento do mundo 
       ``desde que o mundo seja um \textcolor{NordRed}{sistema fechado} (isolado) 
       não sujeito a nenhuma influência causal exterior''
       \begin{itemize}
        \item<7->[-] as grandes leis seguem esse princípio (sistema isolado) 
       \end{itemize}
      \item<8-> Quem estabeleceu que o mundo é ``fechado''?
       \begin{itemize}
        \item<9->[-] se o sistema \textcolor{NordOrange}{não} é fechado, não há 
        problema da \textcolor{NordOrange}{ação particular} de Deus.
       \end{itemize}
     \end{itemize}
     \centering
      \begin{minipage}{\textwidth}
      \begin{block}<10->{Diferença Crucial}
       \centering
       \onslide<11->{\textcolor{NordBrightCyan}{Dizer como as coisas são sempre}}\\
       \onslide<12->{$\neq$}\\ 
       \onslide<13->{\textcolor{NordCyan}{Dizer como as coisas são quando nenhum 
       agente exterior ao universo age.}}
      \end{block}
     \end{minipage}
  \end{itemize}
	
\end{frame}

\begin{frame}{\textbf{A perspectiva newtoniana}}
 \begin{exampleblock}{Relembrando Lógica\ldots}
   \begin{itemize}
    \item[$\bullet$] Tabela Verdade da Conjunção, Disjunção e Condicional\\
     \begin{center}
      \only<2-10>{
       \begin{tabular}{ccccc}
       \toprule
       $p$ & $q$ & $p \wedge q$ & $p \vee q$ & \only<3->{\textcolor{NordYellow}{$p \rightarrow q$}}\\
       \midrule
       \V  & \V  & \V           & \V         & \only<5->{\textcolor{NordYellow}{\V}}\\
       \V  & \F  & \F           & \V         & \only<4->{\textcolor{NordOrange}{\F}}\\
       \F  & \V  & \F           & \V         & \only<5->{\textcolor{NordYellow}{\V}}\\
       \F  & \F  & \F           & \F         & \only<5->{\textcolor{NordYellow}{\V}}\\
       \bottomrule
       \end{tabular}
      }
      \only<11->{
      \begin{tabular}{ccccccc}
      \toprule
      $p$ & $q$ & $p \wedge q$ & $p \vee q$ & \textcolor{NordYellow}{$p \rightarrow q$} & \only<12->{$\sim\! p$ & \only<14->{\textcolor{NordYellow}{$\sim\! p \vee q$}}} \\
      \midrule
      \V  & \V  & \V           & \V         & \textcolor{NordYellow}{\V} & \only<13->{\F} &\only<16->{\textcolor{NordYellow}{\V}}\\
      \V  & \F  & \F           & \V         & \textcolor{NordOrange}{\F} & \only<13->{\F} &\only<15->{\textcolor{NordOrange}{\F}}\\
      \F  & \V  & \F           & \V         & \textcolor{NordYellow}{\V} & \only<13->{\V} &\only<16->{\textcolor{NordYellow}{\V}}\\
      \F  & \F  & \F           & \F         & \textcolor{NordYellow}{\V} & \only<13->{\V} &\only<16->{\textcolor{NordYellow}{\V}}\\
      \bottomrule
      \end{tabular}
      }
     \end{center}
    \item<1, 6->[$\bullet$] Equivalências Lógicas
     \begin{itemize}
      \item<7-> $ p \wedge (q\vee r) \Leftrightarrow (p\wedge q) \vee (p\wedge r) $
      \item<8-9,10-> \only<8-9>{$ p\rightarrow q \Leftrightarrow\, \sim\! p \vee q $} 
                     \only<10->{\textcolor{NordYellow}{$ p\rightarrow q \Leftrightarrow\, \sim\! p \vee q $}}
      \item<9-> $ (p \vee q) \rightarrow r \Leftrightarrow (p\rightarrow r) \wedge (q\rightarrow r)$
     \end{itemize}
   \end{itemize}
 \end{exampleblock}
\end{frame}

\begin{frame}{\textbf{Determinismo necessariamente verdadeiro?}}
	\begin{description}
	 \item<2->[$u$:] universo é casualmente fechado
		\item<3->[$p$:] consequentes de todas as leis
		\item<4->[$\ell$:] passado
		\item<5->[$f$:] futuro
		\item<6->[$\square$:] operador ``necessidade''
	\end{description}
	\centering
	\begin{minipage}{\textwidth}
     \begin{exampleblock}<7->{Definições}
	  \begin{itemize}
		 \item<8-> \textcolor{NordYellow}{Lei Natural}: \only<9->{$u\rightarrow p$}
		 \item<10-> \textcolor{NordYellow}{Determinismo}: \only<11->{$[(u\rightarrow p) \wedge \ell]\rightarrow f$}
		 \item<12-> \textcolor{NordYellow}{Determinismo Necessariamente Verdadeir}: \only<13->{$\square\left\{\,[(u\rightarrow p) \wedge \ell]\rightarrow f\,\right\}$}
	  \end{itemize}
	 \end{exampleblock}
	\end{minipage}
\end{frame}

\begin{frame}{\textbf{Demonstração}}
\onslide<2->{Suponha o Determinismo \textcolor{NordCyan}{necessariamente} verdadeiro.}
\onslide<3->{Então,}
\only<4->{
 \begin{align*}
        \only<4->{\square\left\{\,[(u\rightarrow p) \wedge \ell]\rightarrow f\,\right\}} 
		\only<5->{&\Leftrightarrow}
		\only<6->{\square\left\{\,[\textcolor{NordRed}{(\sim\! u \vee p)} \wedge \ell]\rightarrow f\,\right\}}\\
		\only<7->{&\Leftrightarrow
		\square\left\{\,[\ell \wedge \textcolor{NordRed}{(\sim\! u \vee p)}]\rightarrow f\,\right\}}\\
		\only<8->{&\Leftrightarrow}
		\only<9->{\square\left\{\,[\only<10->{\textcolor{NordYellow}{(\ell \wedge \sim\! u)}} \only<11->{\vee} \only<12->{\textcolor{NordOrange}{(\ell \wedge p)}}]\rightarrow f\,\right\}}\\
		\only<13->{&\Leftrightarrow}
		\only<14->{\square\left\{\,\textcolor{NordYellow}{(\ell \wedge \sim\! u)}\rightarrow f\,\right\}} \only<15->{\quad\wedge\quad} \only<16->{\square\left\{\,\textcolor{NordOrange}{(\ell \wedge p)}\rightarrow f\,\right\}}
 \end{align*}
}
	
	\begin{itemize}
		\item<17->[$\bullet$] $(\ell \wedge \sim\! u ) \rightarrow f$ é falso, pois:
		 \begin{itemize}
		     \item<18-|alert@18> existe um mundo possível com o mesmo passado que o nosso: $\ell$;
			 \item<19-|alert@19> não seja fechado: $\sim\! u$;
			 \item<20-|alert@20> mas, não possui o mesmo futuro: $\sim\! f$.
		 \end{itemize}
		\item<21->[$\bullet$] Como $(\ell \wedge \sim\! u) \rightarrow f$ é falso, então
			$[(u\rightarrow p) \wedge \ell]\rightarrow f$ é necessariamente falso.
			O que é uma contradição!
		\item<22-> A contradição veio do fato de supormos o Determinismo necessariamente
		verdadeiro.
	\end{itemize}
\end{frame}

\begin{frame}{A perspectiva newtoniana}
 \centering
 \begin{minipage}{\textwidth}
  \begin{exampleblock}{Resumo\ldots}
   \begin{itemize}[<+->]
    \item Não  há ``violação'' de lei; pois, ao agir Deus, o sistema não é mais
     fechado;
    \item Não cabe a ciência clássica afirmar que as leis não podem ser violadas;
    \item Portanto,
     \begin{align*}
      \only<4->{\text{Ciência Clássica } &\not\Rightarrow \text{\textcolor{NordYellow}{ Determinismo}}}\\
      \only<5->{&\text{ ou}}\\
      \only<6->{\text{Ciência Clássica } &\not\Rightarrow \text{\textcolor{NordYellow}{ Universo casualmente fechado}}}
     \end{align*}
   \end{itemize}
  \end{exampleblock}
 \end{minipage}
\end{frame}

%------------------------------------------------------------------------------
\subsection{A perspectiva laplaciana}
%------------------------------------------------------------------------------
\begin{frame}{\textbf{A perspectiva laplaciana}}
 \begin{itemize}
  \item<2->[$\bullet$] A perspectiva laplaciana orienta o pensamento dos 
   teólogos da não interferência divina.
  \item<3->[$\bullet$] $\only<4->{\text{Perspectiva laplaciana }} 
   \only<5->{=}
   \only<6->{\text{ newtoniana }} 
   \only<7->{+}
   \only<8->{\text{fecho causal do universo.}} $
    \begin{itemize}
     \item<9-> \textcolor{NordRed}{acréscimo filosófico}
     \item<10-> a ciência clássica não atesta que o ``universo é fechado''
    \end{itemize}
  \item<11->[$\bullet$] $\text{não poder agir } \neq \underbrace{\text{ não agir na prática}}_{\text{perspectiva laplaciana}} $
  \item<12->[$\bullet$] Consequências para a Liberdade Humana
    \begin{itemize}
     \item<13-> nenhuma ação humana é livre
    \end{itemize}
 \end{itemize}
 \centering
 \begin{minipage}{\textwidth}
   \begin{exampleblock}<14->{Conclusão\ldots}
     Não há conflito entre Ciência e Religião!
     \only<15->{
      O conflito existe entre a Religião e uma metafísica particular, 
      na qual afirma que o universo é causalmente fechado.
     }
   \end{exampleblock}
 \end{minipage}
\end{frame}
